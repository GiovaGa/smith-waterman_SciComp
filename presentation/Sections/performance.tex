

\section{Performance analysis}

    \begin{frame}{Test setup}

        We generated test data starting from the genome of an organism found online. We then copied it and modified it.
        We then compared the original and the modified versions. 

        Hardware characteristics:
        \begin{itemize}
        \item CPU: Intel i7-1165G7 (8) @ 2.8GHz
        \item RAM: 8 GB
        \item OS: Arch Linux (I use arch btw)
        \end{itemize}
        
        
    \end{frame}


    \begin{frame}{Performances}
        \begin{figure}[htbp]
            \begin{subfigure}{0.49\linewidth}
            \includegraphics[width=0.95\linewidth]{Sections/plots/Time | CASE: 49.png}
            \end{subfigure}
            \begin{subfigure}{0.49\textwidth}
            \includegraphics[width=0.95\linewidth]{Sections/plots/speedup_48.png}
            \end{subfigure}
            \caption{Running times and speedup plots for each of the algorithms.}
        \end{figure}
        
    \end{frame}


    \begin{frame}{Insights}
        \begin{table}
        \centering
        \begin{tabular}{c|cccc}
            \textbf{Algorithm} & \textbf{1 thread} & \textbf{2 threads}& \textbf{4 threads}& \textbf{8 threads} \\
            \hline
            \textit{Serial} & \\
            \textit{Locks} & \\
            \textit{Tasks} & \\
        \end{tabular}
        \caption{Average running times for a sample input of size X and Y, for different numbers of maximum number of threads.}
        \end{table}
        
        It is interesting to see approximately the same results for each of the approaches. The only notable difference is that the one using locks has a larger variation of the duration between runs.      
        
    \end{frame}